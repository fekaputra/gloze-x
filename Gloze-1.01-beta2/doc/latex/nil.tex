XML differentiates between empty and null data. Such {\itshape nillable\/} elements can be described in the XML schema.


\begin{DoxyCodeInclude}
<?xml version="1.0" encoding="UTF-8"?>
<xs:schema xmlns:xs="http://www.w3.org/2001/XMLSchema" targetNamespace="http://ex
      ample.org/">
        <xs:element name="foobar" type="xs:String" nillable="true"/>
</xs:schema>
\end{DoxyCodeInclude}


An example of a nil element is shown in the example below. A null value is represented in RDF using rdf:nil, the empty list.


\begin{DoxyCodeInclude}
<?xml version="1.0" encoding="UTF-8"?>
<foobar xmlns="http://example.org/" xmlns:xsi="http://www.w3.org/2001/XMLSchema-i
      nstance" 
 xsi:nil="true"/>

\end{DoxyCodeInclude}
 
\begin{DoxyCodeInclude}
@prefix ns2:     <http://example.org/> .
@prefix ns1:     <http://example.org/def/> .
@prefix xs_:     <http://www.w3.org/2001/XMLSchema#> .
@prefix rdf:     <http://www.w3.org/1999/02/22-rdf-syntax-ns#> .
@prefix xs:      <http://www.w3.org/2001/XMLSchema> .

<http://example.org/nil.xml>
      ns2:foobar () .
\end{DoxyCodeInclude}


Compare this with the same element simply left empty.


\begin{DoxyCodeInclude}
<?xml version="1.0" encoding="UTF-8"?>
<foobar xmlns="http://example.org/" />

\end{DoxyCodeInclude}
 
\begin{DoxyCodeInclude}
@prefix ns2:     <http://example.org/> .
@prefix ns1:     <http://example.org/def/> .
@prefix xs_:     <http://www.w3.org/2001/XMLSchema#> .
@prefix rdf:     <http://www.w3.org/1999/02/22-rdf-syntax-ns#> .
@prefix xs:      <http://www.w3.org/2001/XMLSchema> .

<http://example.org/nil1.xml>
      ns2:foobar "" .
\end{DoxyCodeInclude}
 