The XML schema instance namespace defines xsi:type allowing elements be be explicitly annotated with type information. This can be used as an alternative mechanism to substitution groups, but also provides a way to jump back into a schema from within xs:any content. An example of this is shown below using an XML document that has no schema.

The document element 'myLink' is not defined in the schema, so is processed as if it were subject to xs:any and a default no-\/schema mapping is employed. This unidentified element is treated as unqualified and is defined in the default namespace set by the xmlns parameter.

The link schema doesn't define the 'myLink' element, but it does define the 'SimpleLink' content. The XML instance refers to this using an xsi:type attribute. The RDF mapping for this generates a corresponding rdf:type statement. The content of 'myLink' is then processed according to the content model of 'SimpleLink', so we have escaped from the xs:any no-\/schema mapping.


\begin{DoxyCodeInclude}
<?xml version="1.0" encoding="UTF-8"?>
<myLink xmlns:eg="http://example.org/" xmlns:xlink="http://www.w3.org/1999/xlink"
      
        xmlns:xsi="http://www.w3.org/2001/XMLSchema-instance" xsi:type="eg:Simple
      Link"
        xlink:href="foo.xml" />

\end{DoxyCodeInclude}
 
\begin{DoxyCodeInclude}
<?xml version="1.0" encoding="UTF-8"?>
<xs:schema xmlns:xs="http://www.w3.org/2001/XMLSchema"
        targetNamespace="http://example.org/" xmlns:xlink="http://www.w3.org/1999
      /xlink">
        <xs:import namespace="http://www.w3.org/1999/xlink" schemaLocation="xlink
      .xsd" />

        <xs:complexType name="SimpleLink">
                <xs:attributeGroup ref="xlink:simpleLink"/>
        </xs:complexType>

</xs:schema>
\end{DoxyCodeInclude}


The use of the 'SimpleLink' content model is evidenced by the correct datatyping of the xlink:href and the insertion of the xlink:type implied by the xlink attribute group.


\begin{DoxyCodeInclude}
@prefix ns2:     <http://example.org/> .
@prefix ns1:     <http://example.org/def/> .
@prefix eg:      <http://example.org/> .
@prefix xlink_:  <http://www.w3.org/1999/xlink#> .
@prefix xs_:     <http://www.w3.org/2001/XMLSchema#> .
@prefix xlink:   <http://www.w3.org/1999/xlink> .
@prefix rdf:     <http://www.w3.org/1999/02/22-rdf-syntax-ns#> .
@prefix xs:      <http://www.w3.org/2001/XMLSchema> .

<http://example.org/linkType.xml>
      ns1:myLink
              [ a       ns2:SimpleLink ;
                xlink_:href "http://example.org/foo.xml"^^xs_:anyURI ;
                xlink_:type "simple"^^xs_:string
              ] .
\end{DoxyCodeInclude}


\begin{DoxySeeAlso}{See also}
\href{http://www.w3.org/2001/XMLSchema-instance}{\tt http://www.w3.org/2001/XMLSchema-\/instance} 
\end{DoxySeeAlso}
